\section{Related Work}
\label{sec:rel}
In [1] recommendation of tags is done automatically by training a classifier that maps audiovisual features from millions of you-tube video to supplied tags by an uploader of video. The system learns a vocabulary of tags and suggests tags which are relevant to the video. A detailed analysis of usefulness of comment is presented in [2] including influence of comment sentiment on rating of the comment using SentiWordNet theasures. They also predict the community acceptance of an unrated comment using svm classification and term based representation  of comments to categorize a comment as \textit{accepted} or \textit{not accepted}. Other work on sentiment classification and opinion mining [3] deals with assigning positive, negative or neutral sentiment to a text using text orientation and linguistic features. 
In our project we do sentiment analysis using four different classification models. In addition to this we do Category prediction based on comments and tags.
[1]