\section{Conclusion}
\label{sec:conclusion}
We applied classification techniques to the problems of sentiment and category classification on a sample dataset of popular YouTube videos.
We observed that, using TF-IDF features, the sentiment of YouTube comments can be estimated with high precision. Naive Bayes classifier showed the lowest accuracy among the $4$ models we tried. This result conforms with our expectation as Naive Bayes independence assumption is not satisfied by TF-IDF features. Multinomial Logistic Regression (Softmax regression), Ridge classifier and Linear SVC all had acceptable prediction accuracy with Linear SVC outperforming the other models. It was observed that Naive Bayes is not well-calibrated over the range $[0,1]$ while Logistic Regression is. It is also better calibrated than the other models in general. Precision-Recall curves were plot and we observed that Precision and Recall have an inverse relation. We examined the effect of dimensionality reduction on TF-IDF features input to the aforementioned classification techniques. Increasing the number of dimensions of the projection space enhances accuracy. However, for the YouTube dataset, the accuracy achieved using dimensionality reduction is still far from the accuracy achieved with no dimensionality reduction. This shows that with the highest dimension of the projection space we could try, being $175$, there is still much information in the original data space which cannot be captured in the projection space. We showed that the accuracy of classifiers decreases with the number of categories. This is because choosing the right class becomes more difficult for higher number of classes.

We then considered the problem of category prediction using comment TF-IDF features and video tags. We observed that video categories can be estimated with higher accuracy using tags than comment features. This is because tags are keywords explicitly chosen relevant to the content of videos while comment features are noisy and may not always be indicative of video categories. Despite, both methods can lend themselves well to different scenarios depending on the data at hand. Comment features can be used if the association between the videos and the comments is lost due to reasons like loss of database while tags can be used in ordinary situations when enough metadata is available about videos. Similar to sentiment analysis, Linear SVC had the best performance in category classification.

 